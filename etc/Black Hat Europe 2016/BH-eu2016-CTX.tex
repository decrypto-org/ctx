\documentclass[a4paper, 11 pt, conference]{article}  % Comment this line out  if you need a4paper

% The following packages can be found on http:\\www.ctan.org
\usepackage{graphics} % for pdf, bitmapped graphics files
\usepackage{url}
\usepackage{listings}

\title{\textbf{CTX: Eliminating BREACH with Context Hiding}}

\author{
    Dimitris Karakostas\footnotemark[1]\\
    University of Athens\\
    dimit.karakostas@gmail.com\\
    \and
    Aggelos Kiayias\footnotemark[1]\\
    University of Edinburgh\\
    akiayias@inf.ed.ac.uk\\
    \and
    Eva Sarafianou\footnotemark[1]\\
    University of Athens\\
    eva.sarafianou@gmail.com\\
    \and
    Dionysis Zindros\thanks{Research supported by ERC project CODAMODA, project \#259152.}\\
    University of Athens\\
    dionyziz@di.uoa.gr\\
}

\date{}
\pagenumbering{arabic}

\begin{document}

\maketitle
\thispagestyle{plain}
\pagestyle{plain}

\begin{abstract}

The BREACH attack presented at Black Hat USA 2013 has still not been mitigated,
even in the latest versions of TLS, despite the new developments and
optimizations presented at Black Hat Asia 2016. BREACH and similar attacks pose
a threat against all practical web applications which use compression together
with encryption. We present a generic defense method which eliminates problems
that arise from compression detectability features of existing protocols. We
introduce CTX, Context Transformation Extension, a cryptographic method which
defends against BREACH, CRIME, TIME, and any compression side-channel attack in
general. CTX operates at the application layer and uses context hiding in a
per-origin manner to separate secrets from different origins in order to avoid
cross-compressibility.

\end{abstract}

\section{Introduction}
In 2012 CRIME~\cite{c1} showed for the first time that side-channel compression
attacks can be successful against TLS. CRIME targeted HTTPS requests and has
since been mitigated by disabling compression at the TLS level~\cite{c2}.

In 2013 TIME~\cite{c3} and BREACH~\cite{c4} introduced an attack vector that exploited
compression on HTTP responses to compromise TLS. This vector takes advantage of the
characteristics of the DEFLATE algorithm~\cite{c5}, the basis of most
compression applications, in order to steal secrets from applications using
stream ciphers.

In 2015 RC4 is considered insecure~\cite{c6} which forces most websites to use
AES block ciphers. Services like Facebook also tried to prevent BREACH~\cite{c7}
using secret masking, although this method protected CSRF tokens only and the
fundamental aspects of BREACH were still not mitigated.

In 2016, both Rupture~\cite{c8} and HEIST~\cite{c9} introduced new threats
regarding compression side-channel attacks. Rupture showed that BREACH can
evolve to attack major web applications and steal secrets that were not
previously considered as targets of BREACH. It also incorporated statistical
methods to bypass noise induced from block ciphers or random data included in the response plaintext.

HEIST demonstrated that compression-based attacks, such as CRIME and BREACH, can
be performed solely in the browser by a malicious website or script. It does not
require Man-in-the-Middle agents since it abuses the way responses are sent at the TCP level.

These attack techniques, which pose an imminent threat to online security and privacy, have still not been mitigated.

Our work introduces a generic defense method which disqualifies compression
detectability features of existing protocols. CTX is a cryptographic method
which defends against any compression side-channel attack. It prevents
cross-compressibility by separating the secrets from different origins and using
context hiding in a per-origin manner.

The existing suggested defense for BREACH~\cite{c10} includes disabling
compression or completely bypassing compression, which results in significant
performance penalties. On the other hand, there has not been proposed a solution
that keeps compression intact and solves the security issues. It is not known if
such a solution is even possible. Our method lies between the two options
regarding compression usage. We achieve a good balance by slightly reducing
compression size and time performance while achieving full security.

We release an open source implementation of CTX in popular web frameworks both
for client-side and server-side web applications. Our implementation runs at the
application layer, is opt-in, and does not require modifications to web
standards or the underlying web server.

We conclude that if secrets are separated by origin at the application level using the CTX defense, compression side-channel attacks are mitigated.

\begin{thebibliography}{10}

\bibitem{c1} J. Rizzo, T. Duong: The CRIME attack, Ekoparty, 2012

\bibitem{c2} D. Goodin, Crack in Internet’s foundation of trust allows HTTPS session hijacking, Ars Technica, 2012

\bibitem{c3} Y. Gluck, N. Harris, A. Prado, BREACH: Reviving the CRIME attack, Black Hat USA, 2013

\bibitem{c4} [online] URL: \url{https://en.wikipedia.org/wiki/DEFLATE} 

\bibitem{c5} A.Popov, Prohibiting RC4 Cipher Suites, RFC 7465, 2015

\bibitem{c6} [online] URL: \url{https://www.facebook.com/notes/protect-the-graph/preventing-a-breach-attack/1455331811373632}

\bibitem{c7}  D.Karakostas, D.Zindros: Practical New Developments in the BREACH Attack, Black Hat Asia, 2016

\bibitem{c8} M. Vanhoef, Tom Van Goethem: HEIST: HTTP Encrypted Information can be 220 Stolen through TCP-windows, Black Hat USA 2016

\end{thebibliography}

\end{document}
