\documentclass[a4paper, 11 pt, conference]{article}  % Comment this line out  if you need a4paper

% The following packages can be found on http:\\www.ctan.org
\usepackage{graphics} % for pdf, bitmapped graphics files
\usepackage{url}
\usepackage{listings}

\title{\textbf{CTX: Eliminating BREACH with Context Hiding}}

\author{
    Dimitris Karakostas\footnotemark[1]\\
    University of Athens\\
    dimit.karakostas@gmail.com\\
    \and
    Aggelos Kiayias\footnotemark[1]\\
    University of Edinburgh\\
    akiayias@inf.ed.ac.uk\\
    \and
    Eva Sarafianou\footnotemark[1]\\
    University of Athens\\
    eva.sarafianou@gmail.com\\
    \and
    Dionysis Zindros\thanks{Research supported by ERC project CODAMODA, project \#259152.}\\
    University of Athens\\
    dionyziz@di.uoa.gr\\
}

\date{}
\pagenumbering{arabic}

\begin{document}

\maketitle
\thispagestyle{plain}
\pagestyle{plain}

\begin{abstract}

The BREACH attack presented at Black Hat USA 2013 has still not been mitigated,
even in the latest versions of TLS, despite the new developments and
optimizations presented at Black Hat Asia 2016. BREACH and similar attacks pose
a threat against all practical web applications which use compression together
with encryption. We present a generic defense method which eliminates problems
that arise from compression detectability features of existing protocols. We
introduce CTX, Context Transformation Extension, a cryptographic method which
defends against BREACH, CRIME, TIME, and any compression side-channel attack in
general. CTX operates at the application layer and uses context hiding in a
per-origin manner to separate secrets from different origins in order to avoid
cross-compressibility.

\end{abstract}

\end{document}
